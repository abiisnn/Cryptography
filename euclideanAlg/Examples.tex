% Definición tipo de documento
\documentclass[12pt,openany]{book}

% Paquetes
\usepackage{lmodern}
\usepackage[T1]{fontenc}
\usepackage[spanish,activeacute]{babel}
\usepackage{mathtools}
\usepackage{amsmath}
\usepackage{amssymb}
\usepackage[latin1]{inputenc}

%Para hacer cajas bonitas
\usepackage{fancybox}

%Permite crear columnas en el documento
\usepackage{multicol} 
\usepackage{color}
\usepackage{comment}

%Permite incluir imagenes
\usepackage{graphicx}
\graphicspath{{images/}} %Indicamos la ruta de las imagenes

% Configuracion de encabezado y pie de pagina
\usepackage{fancyhdr}                                          
    \pagestyle{fancy}                                               
    \setlength{\headheight}{16pt}                                   
    \setlength{\parskip}{0.5em}                                    
    \renewcommand{\footrulewidth}{0.5pt}                            
    \lhead{ }

\title{An\'alisis Vectorial}
\author{Nico\'as Sayago Abigail}

% ////////////////////////////////////////
% ///////// INICIO DEL DOCUMENTO /////////
%/////////////////////////////////////////

\begin{document}
    \begin{titlepage}
        \centering
        \rule{\linewidth}{0.5mm} \\[1.0cm]
            { \huge \bfseries Inverse multiplicative}\\[1.0cm] 
        \rule{\linewidth}{0.5mm} \\[2.0cm]
        \centering
        Abigail Nicol\'as Sayago 
    \end{titlepage}
   

\tableofcontents

\newpage

	With $8^{-1}$ mod $13$
	
	\begin{itemize}
		\item Obtain each operation by Euclidean Algorithm
			\begin{itemize}
				\item[\Checkmark] $8 = 13 * 0 + 8$
				\item[\Checkmark] $13 = 8 * 1 + 5$
				\item[\Checkmark] $8 = 5 * 1 + 3$
				\item[\Checkmark] $5 = 3 * 1 + 2$
				\item[\Checkmark] $3 = 2 * 1 + 1$
				\item[\Checkmark] $2 = 1 * 2 + 0$	
			\end{itemize}

		\item Second
			\begin{itemize}
				\item[\Checkmark] $13 = 8 * 1 + 5$ $\longrightarrow$ $5 = 13 + 8(-1)$
				
				\item[\Checkmark] $8 = 5 * 1 + 3$ $\longrightarrow$ $3 = 8 + 5(-1)$
				
				\item[\Checkmark] $5 = 3 * 1 + 2$ $\longrightarrow$ $2 = 5 + 3(-1)$
				
				\item[\Checkmark] $3 = 2 * 1 + 1$ $\longrightarrow$ $1 = 3 + 2(-1)$
			\end{itemize}

		\item Th
			\begin{itemize}
				
				\item[\Checkmark] $3 = 8 + 5(-1)$
					\begin{equation*}
						\begin{split}
							3 &= 8 + 5(-1)	\\
							  &= 8 + \left[ 5 = 13 + 8(-1) \right](-1)	\\
							  &= 8 + 13(-1) + 8(1) \\
							  &= 8(2) + 13(-1)	\\
						\end{split}
					\end{equation*}

				\item[\Checkmark] $5 = 13 + 8(-1)$
				
				
				\item[\Checkmark] $2 = 5 + 3(-1)$
				
				\item[\Checkmark] $1 = 3 + 2(-1)$
			\end{itemize}


			
	\end{itemize}
			Sustituimos en la ecuaci\'on diferencial:
				\begin{equation*}
					\begin{split}
						\vec{C}m^{2}e^{mt}+2\alpha\vec{C}me^{mt}+w^{2}\vec{C}e^{mt}=\vec{0}
					\end{split}
				\end{equation*}
				Factorizamos:
				\begin{equation*}
					\begin{split}
						m_{1,2}&=\frac{-b\pm\sqrt{b^{2}--4ac}}{2a} 		 		\\
							   &=\frac{-2\alpha\pm\sqrt{4\alpha^{2}-4w^{2}}}{2}	\\
							   &=-\alpha\pm\sqrt{\alpha^{2}-w^{2}} 
					\end{split}
				\end{equation*}

		% ///////////////////////////
		% /////// Problemas ////////
		% /////////////////////////
		\subsection{Problemas}
			\subsubsection{Ejercicio 1}
				\noindent\textsl{Soluci\'on:}	\\
				\begin{equation*}
					\begin{split}
						\frac{\partial d}{\partial x}x^{2}\ln(3ye^{z})=\ln(3ye^{z})2x
					\end{split}
				\end{equation*}

\end{document}